\documentclass[11pt]{scrartcl}
\usepackage[utf8]{inputenc}
\usepackage{amsmath, amssymb, amsthm, bbm}
\usepackage{booktabs, verbatim, graphicx, framed}
\usepackage[sexy, hints]{evan}
\title{Chapter 6 Homework}
\author{Anay Aggarwal}

\begin{document}

\maketitle
\begin{example}
  [Context Rich Problem]
  Moving stuff across a pond.
\end{example}
\begin{soln}
  To solve this problem, we need to know about work and energy. Specifically,
  $$W_{net}=\Delta E$$
  $$K=\frac12 mv^2$$
  $$W=F\cdot x$$
  The diagram is as follows:
  \begin{center}\includegraphics[scale=0.3]{Work.png}\end{center}
  The system is the rock and the pond. Notice that at the beginning, there is zero kinetic energy since the object is at rest, and zero potential energy.
  At the end, there is only kinetic energy (potential energy is in the wrong direction), so the total energy is $\frac{1}{2}mv_{final}^2$. Hence $\Delta E=\frac{1}{2}mv_f^2$. Noting that
  $$W_{net}=\Delta E$$
  we can compute $\Delta E$, and, in turn, $v_f$, if we can compute $W_{net}$. Notice that
  $$W_{net}=\sum W=\sum F\cdot x$$
  Which is known. Hence
  $$\frac{1}{2}mv_f^2=\sum F\cdot x$$
  $$v_f=\sqrt{\frac{2\sum F\cdot x}{m}}$$
  Before plugging in $F_1=200N, x_1=12m, F_2=500N, x_2=12m, m=150kg$, we can do some sense-making. Notice that the units of $v_f$ are
  $$\sqrt{\frac{kg\frac{m^2}{s^2}}{kg}}=\frac{m}{s}$$
  Which makes sense. Additionally, when $m$ gets large, $v_f$ gets small, which also makes sense. Plugging in, we get
  $$v_f=\sqrt{\frac{2\cdot (200\cdot 12+500\cdot 12)}{150 kg}}=\sqrt{\frac{16800}{150}}\approx 10.58 m/s$$
  Which also makes sense.
\end{soln}
\begin{example}
  [Explanation Task]
  Ranking Work
\end{example}
\begin{soln}
  Suppose pushing a block of mass $m$ with a force $F_0$ on the time interval $[t_1,t_2]$
  pushes it a distance $d_0$. Then the total work on each block in experiment
  $1$ is $F_0\cdot d_0$. So the total work is $2F_0\cdot d_0$. This is the same in experiment two (since work is force \textit{dotted} with distance)
  In experiment 3, the distance will be a bit less than $d_0$ so the work will be a bit less than $2F_0\cdot d_0$, but still positive.
  In experiment 4, the distance for the right block will be larger than $d_0$, so the total work will be larger than $2F_0\cdot d_0$.
  In experiment 5, the blocks won't move so zero net work will be done. Hence:
  $$\exp 5<\exp 3<\exp 1=\exp 2<\exp 4$$
\end{soln}
\end{document}
